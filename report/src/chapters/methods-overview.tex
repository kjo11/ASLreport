\chapter{Methods}

\section{Overview}

The mapping and localization algorithm are both based on the minimization of
photometric errors.

The idea behind photometric error minimization for localization and mapping is
that when the position of the test points with respect to the camera
positions is correct, the image intensities of this point in both cameras are
exactly the same and correspond to the actual color of that point in object
space. 
If the latter is not the case, the relative position of the map with respect to 
the cameras has to be adjusted by tuning the height of the point (mapping) or 
the position of the cameras (localization).

Mappings from object points or camera positions to the camera model are highly
non-linear because of the nature of (pinhole) camera projections and the
non-linear image intensity function (\ref{sec:methods-mapping-photometric} and
\ref{sec:methods-localization-photometric}). The resulting minimzation problem
is therefore non-linear (\ref{sec:methods-mapping-optimization} and
\ref{sec:methods-localization-optimization} convergence difficulties such as
local minima and therefore occur, which yield the necessity of some tricks
(\ref{sec:methods-mapping-schemes} and \ref{sec:methods-localization-schemes}).
